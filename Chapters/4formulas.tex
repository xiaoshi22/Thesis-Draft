
\chapter{Formulas}
\section{Complexity = 3}

We start from the case that complexity is $3$. 

Let a symmetric critical arcgon $\mathcal{F_3}$ be given. Let the base circles of $\mathcal{F}_3$ be $O_c', O_o$ and $O_c$. Since we will calculate the ratio between paths and absolute distance, the exact size of each circle does not matter. Without loss of generality, suppose that the muddle base circle, $O_o$, is a unite circle. That is, $o = (0, 0)$ and the radius of $O_o$ is $1$. Figure~\ref{fig:f1_1}(a) shows how we denote points in $\mathcal{F}_3$. 

\input{PST/f1_1.tex}

Figure~\ref{fig:f1_1}(b) gives an illustration on arcs. Since $\mathcal{F}_3$ is symmetric, we know that 
\[A' = A\text{, } B'= B\text{, }D' = D \text{, }\arc{a'n} = \arc{na}.\]
Also, paths from $p$ to $q$ in $\mathcal{F}_3$ could be
\[A'O_{up}A\text{, }A'D'DA\text{, } A'D'B\text{, }B'DA\text{, } B'B.\]
Since $\mathcal{F}_3$ is critical, all paths listed above have the same length.
Let $\|\overline{sm}| = |\overline{am}| = x$ and $|\overline{cm}| = k$. 
\begin{theorem}
$\rho_{\mathcal{F}_3}(p, q) =  \cfrac{x + \sqrt{k^2+x^2}\cdot\tan^{-1}{\frac{x}{k}}}{\left(\sqrt{1- x^2} - k\right)\cdot\cos{x} + \sqrt{k^2 + x^2}\cdot\cos{\left(\frac{x}{\sqrt{k^2 + x^2}} - x\right)}}.$
\end{theorem}

\begin{proof}

[Goal 1: Find a solution of $x$.]


Let $r$ be the radius of $O_c$ and $O_c'$. That is, $r = \sqrt{x^2+k^2}$.

First, we can show that $\arc{na} = 2x$. Since $A'O_{up}A$ and 
$A'D'DA$ are equal in length, we have
\begin{align*}
    |O_{up}| &= |D'D| \\
    |\arc{a'n}|+|\arc{na}| &= |D'|+|D|.
\end{align*}
Also, since $|\arc{a'n}|=|\arc{na}|$ and $|D'|=|D|$, then
\begin{align*}
    2|\arc{na}| &= 2|D|, \\
    |\arc{na}|&= |D| \\
    &=|\overline{am}|+|\overline{ms}|\\
    &=x+x\\
    &=2x.
\end{align*}

Consider $\arc{aws}$. Since $|\overline{am}| = x$ and  $|\overline{oa}| = 1$, we know $\angle aom = \sin^{-1}{x}$.
Thus,
\begin{align*}
    \arc{aws} &= 2 (1\cdot\angle aom)= 2\cdot \sin^{-1}{x}.
\end{align*}
Then, since $\arc{an}$ and $\arc{aws}$ construct a semicircle, we have
\begin{align*}
    |\arc{\overline{an}}+\arc{aws}| &= \pi \\
    2x +2\sin^{-1}{x} &= \pi\\
    x +\sin^{-1}{x} - \frac{\pi}{2} & = 0.\\
\end{align*}
Therefore,  $x$ is the solution of $x +\sin^{-1}{x} - \frac{\pi}{2}= 0$. The approximate value is $0.739085$.



\vspace{1cm}
[Goal 2: Calculate $P_{\mathcal{F}_3}(p, q)$.]
\begin{figure}[ht]

\psset{unit=.7pt}
\psset{labelsep=7pt}
\begin{center} \small
\begin{pspicture}(-140,-150)(140,150)
    \psline{->}(-130,0)(130,0)
    \psline{->}(0, -130)(0, 130)
    \pscircle(0,0){100}
     \uput[45](0, 0){$o$}
     \psdot(0, 100)
      \uput[135](0, 100){$n$}
       \psdot(0, -100)
     \uput[-135](0, -100){$s$}
     \psdot(100, 0)
      \uput[225](100, 0){$e$}
      \psdot(-100, 0)
     \uput[-45](-100, 0){$w$}
     
    \psarc[linecolor=lightgray, linewidth=2pt](0,0){100}{-90}{8}
    % diagonal
    \psdot(99.268, 13.917)
    \uput[45](99.268, 13.917){$a$}
    \psline(0, -100)(99.268, 13.917)
    % center -> side circle
    \psdot(31.154, -27.082)
    \uput[90](31.154, -27.082){$c$}
    \pscircle(31.154, -27.082){79.27}
    \psarc[linecolor=lightgray, linewidth=2pt](31.154, -27.082){79.27}{31}{246.8}
    
    \psline[linestyle=dashed](0, 0)(99.268, 13.917)%oa
    \psline[linestyle=dashed](31.154, -27.082)(99.268, 13.917)%ca
    \psline[linestyle=dashed](0, 0)(31.154, -27.082)%oc
    
    
     \psdot(49.343, -43.318)
     \uput[-45](49.343, -43.318){$m$}
     \psline[linestyle=dashed](31.154, -27.082)(49.343, -43.318)%cm
     
     \psdot(109.404, -14.408)
      \uput[0](109.404, -14.408){$q$}
    
    
    \psarc[linecolor=lightgray, linewidth=2pt](0,0){100}{172}{-90}
    % diagonal
    % \psdot(-99.268, 13.917)
    % \uput[135](-99.268, 13.917){$a'$}
    \psline(0, -100)(-99.268, 13.917)
    % center -> side circle
    % \psdot(-31.154, -27.082)
    % \uput[90](-31.154, -27.082){$c'$}
    \pscircle(-31.154, -27.082){79.27}
    \psarc[linecolor=lightgray, linewidth=2pt](-31.154, -27.082){79.27}{-66.8}{149}
    %  \psdot(-49.343, -43.318)
    %  \uput[225](-49.343, -43.318){$m'$}
    %  \psline[linestyle=dashed](-31.154, -27.082)(-49.343, -43.318)%cm
     \psdot(-109.404, -14.408)
      \uput[180](-109.404, -14.408){$p$}
    
    % \psline[linecolor=red, linestyle= dashed](-109.404, -14.408)(109.404, -14.408)
    
    \uput[-90](0,-120){(a)}
\end{pspicture}
\psset{unit=.7pt}
\begin{pspicture}(-140,-150)(140,150)
    \psline{->}(-130,0)(130,0)
    \psline{->}(0, -130)(0, 130)
    \pscircle(0,0){100}
     \uput[45](0, 0){$o$}
     \psdot(0, 100)
      \uput[135](0, 100){$n$}
       \psdot(0, -100)
     \uput[-135](0, -100){$s$}
     \psdot(100, 0)
      \uput[225](100, 0){$e$}
      \psdot(-100, 0)
     \uput[-45](-100, 0){$w$}
     
    \psarc[linecolor=lightgray, linewidth=2pt](0,0){100}{-90}{8}
    % diagonal
    \psdot(99.268, 13.917)
    \uput[45](99.268, 13.917){$a$}
    
    
    \psline(0, -100)(99.268, 13.917)
    % center -> side circle
    \psdot(31.154, -27.082)
    \uput[90](31.154, -27.082){$c$}
    \pscircle(31.154, -27.082){79.27}
    \psarc[linecolor=lightgray, linewidth=2pt](31.154, -27.082){79.27}{31}{246.8}
    
    \psline[linestyle=dashed](0, 0)(99.268, 13.917)%oa
    \psline[linestyle=dashed](31.154, -27.082)(99.268, 13.917)%ca
    \psline[linestyle=dashed](0, 0)(31.154, -27.082)%oc
    
    
     \psdot(49.343, -43.318)
     \uput[-45](49.343, -43.318){$m$}
     \psline[linestyle=dashed](31.154, -27.082)(49.343, -43.318)%cm
     
     \psdot(109.404, -14.408)
      \uput[0](109.404, -14.408){$q$}
    
    
    % \psarc[linecolor=lightgray, linewidth=2pt](0,0){100}{172}{-90}
    % diagonal
    % \psdot(-99.268, 13.917)
    % \uput[135](-99.268, 13.917){$a'$}
    % \psline(0, -100)(-99.268, 13.917)
    % center -> side circle
    % \psdot(-31.154, -27.082)
    % \uput[90](-31.154, -27.082){$c'$}
    % \pscircle(-31.154, -27.082){79.27}
    % \psarc[linecolor=lightgray, linewidth=2pt](-31.154, -27.082){79.27}{-66.8}{149}
    %  \psdot(-49.343, -43.318)
    %  \uput[225](-49.343, -43.318){$m'$}
    %  \psline[linestyle=dashed](-31.154, -27.082)(-49.343, -43.318)%cm
    %  \psdot(-109.404, -14.408)
      \uput[180](-109.404, -14.408){$p$}
    % 
    % \psline[linecolor=red, linestyle= dashed](-109.404, -14.408)(109.404, -14.408)
    
    \uput[45](71, 71){$2x$}
    \uput[-45](103, -103){$2r\cdot \tan^{-1}\frac{x}{k}$}
    
    \uput[-90](0,-120){(b)}
\end{pspicture}

\caption[Calculate the path distance from $p$ to $q$. ]{Calculate the path distance from $p$ to $q$.}\label{fig:f1_2}
\end{center}
\end{figure} 


According to Figure~\ref{fig:f1_2}, %??????f1-2
$A'O_{up}A$ and $B'B$ are two path from $p$ to $q$. Then, we have
\begin{align*}
    2 P_{\mathcal{F}_3}(p, q) & = |A'O_{up}A|+|B'B|\\
    P_{\mathcal{F}_3}(p, q) & = \frac{|A'O_{up}A|}{2}+\frac{|B'B|}{2}\\
    &= |\arc{na}|+|A|+|B|\\
    &=|\arc{na}|+|\arc{aqs}|.
\end{align*}
We have shown that $|\arc{na}| = 2x$. Also, we know that $|\arc{aqs}| = 2(r\cdot \angle acm)$, and $\angle acm = \tan^{-1}{\frac{x}{k}}$. Thus,
\[|\arc{aqs}| = 2r\cdot \tan^{-1}{\frac{x}{k}}.\]
Therefore, 
\begin{align*}
    P_{\mathcal{F}_3}(p, q) &=|\arc{na}|+|\arc{aqs}|\\
    &=2x + 2r\cdot \tan^{-1}{\frac{x}{k}}.
\end{align*}


\vspace{1cm}
[Goal 3: Calculate $D_{\mathcal{F}_3}(p, q)$.]
\begin{figure}[ht]

\psset{unit=.7pt}
\psset{labelsep=7pt}
% \begin{center} \small
\begin{pspicture}(-140,-150)(140,150)
    \psline{->}(-130,0)(130,0)
    \psline{->}(0, -130)(0, 130)
    \pscircle(0,0){100}
     \uput[45](0, 0){$o$}
     \psdot(0, 100)
      \uput[135](0, 100){$n$}
       \psdot(0, -100)
     \uput[-135](0, -100){$s$}
     \psdot(100, 0)
      \uput[225](100, 0){$e$}
      \psdot(-100, 0)
     \uput[-45](-100, 0){$w$}
     
    \psarc[linecolor=lightgray, linewidth=2pt](0,0){100}{-90}{8}
    % diagonal
    \psdot(99.268, 13.917)
    \uput[45](99.268, 13.917){$a$}
    \psline(0, -100)(99.268, 13.917)
    % center -> side circle
    \psdot(31.154, -27.082)
    \uput[90](31.154, -27.082){$c$}
    \pscircle(31.154, -27.082){79.27}
    \psarc[linecolor=lightgray, linewidth=2pt](31.154, -27.082){79.27}{31}{246.8}
    
     \psdot(49.343, -43.318)
     \uput[-45](49.343, -43.318){$m$}
     \psline[linestyle=dashed](31.154, -27.082)(49.343, -43.318)%cm
     
     \psdot(109.404, -14.408)
      \uput[0](109.404, -14.408){$q$}
    
    
    \psarc[linecolor=lightgray, linewidth=2pt](0,0){100}{172}{-90}
    % diagonal
    \psdot(-99.268, 13.917)
    \uput[135](-99.268, 13.917){$a'$}
    \psline(0, -100)(-99.268, 13.917)
    % center -> side circle
    \psdot(-31.154, -27.082)
    \uput[90](-31.154, -27.082){$c'$}
    \pscircle(-31.154, -27.082){79.27}
    \psarc[linecolor=lightgray, linewidth=2pt](-31.154, -27.082){79.27}{-66.8}{149}
     \psdot(-49.343, -43.318)
     \uput[225](-49.343, -43.318){$m'$}
     \psline[linestyle=dashed](-31.154, -27.082)(-49.343, -43.318)%cm
     \psdot(-109.404, -14.408)
      \uput[180](-109.404, -14.408){$p$}
    
    \psline[linecolor=red, linestyle= dashed](-109.404, -14.408)(109.404, -14.408)
    
    \uput[-90](0,-120){(a) . }
\end{pspicture}
\psset{unit=.7pt}
\begin{pspicture}(-140,-150)(140,150)
    \psline{->}(-130,0)(130,0)
    \psline{->}(0, -130)(0, 130)
    \pscircle(0,0){100}
    %  \uput[45](0, 0){$o$}
    %  \psdot(0, 100)
    %   \uput[135](0, 100){$n$}
     \psdot(0, -100)
     \uput[-135](0, -100){$s$}
    %  \psdot(100, 0)
    %   \uput[225](100, 0){$e$}
    %   \psdot(-100, 0)
    %  \uput[-45](-100, 0){$w$}
     \uput[45](0, 100){$O_{up}$}
 
    \psarc[linecolor=lightgray, linewidth=2pt](0,0){100}{-90}{8}
    % diagonal
    \psdot(99.268, 13.917)
    \uput[45](99.268, 13.917){$a$}
    
    \psline(0, -100)(99.268, 13.917)
    % center -> side circle
    % \psdot(31.154, -27.082)
    % \uput[90](31.154, -27.082){$c$}
    \pscircle(31.154, -27.082){79.27}
    \psarc[linecolor=lightgray, linewidth=2pt](31.154, -27.082){79.27}{31}{246.8}
    
    %  \psdot(49.343, -43.318)
      \uput[-45](49.343, -43.318){$D$}
      \uput[225](98.646, -86.636){$B$}
    %  \psline[linestyle=dashed](31.154, -27.082)(49.343, -43.318)%cm
     
     \psdot(109.404, -14.408)
      \uput[0](109.404, -14.408){$q$}
       \uput[45](104.336,-0.2455){$A$}%a q mid point
    
    
    \psarc[linecolor=lightgray, linewidth=2pt](0,0){100}{172}{-90}
    % diagonal
    \psdot(-99.268, 13.917)
    \uput[135](-99.268, 13.917){$a'$}
    \psline(0, -100)(-99.268, 13.917)
    % center -> side circle
    % \psdot(-31.154, -27.082)
    % \uput[90](-31.154, -27.082){$c'$}
    \pscircle(-31.154, -27.082){79.27}
    \psarc[linecolor=lightgray, linewidth=2pt](-31.154, -27.082){79.27}{-66.8}{149}
    %  \psdot(-49.343, -43.318)
     \uput[225](-49.343, -43.318){$D'$}
     \uput[-45 ](-98.646, -86.636){$B'$}
    %  \psline[linestyle=dashed](-31.154, -27.082)(-49.343, -43.318)%cm
     \psdot(-109.404, -14.408)
      \uput[180](-109.404, -14.408){$p$}
       \uput[135](-104.336,-0.2455){$A'$}%a q mid point
    
    
    % \psline[linecolor=red, linestyle= dashed](-109.404, -14.408)(109.404, -14.408)
 \uput[-90](0,-120){(b) Example 2. }
\end{pspicture}

\caption{
}\label{fig:f1_3}
% \end{center}
\end{figure} 


Let $\overline{oc_x}$ and $\overline{cq_x}$ be the images of $\overline{oc}$ and $\overline{cq}$ in positive $x$ direction respectively. Then, we have
\[\frac{D_{\mathcal{F}_3}(p, q)}{2} = |oc_x|+|cq_x|.\]

Consider $|\overline{oc_x}|$. Since $|\overline{xs}| = x$, we have $\angle soc = \sin^{-1}{x}$. Then,
\begin{align*}
    \angle c_xoc &=\angle c_xos + \angle soc\\
    &=-\frac{\pi}{2}+\sin^{-1}x.
\end{align*}
Since we have shown that $x+\sin^{-1}x-\frac{\pi}{2} = 0$, we know
\[\angle c_xoc = -x.\]
Then, since $|\overline{om}| = \sqrt{|\overline{oa}|^2-|\overline{am}|^2} = \sqrt{1-x^2}$, and we defined $|\overline{cm}| = k$, 
\begin{align*}
    |\overline{oc}| &= |\overline{om}|-|\overline{cm}|\\
    &= \sqrt{1-x^2} - k.
\end{align*}
Therefore, 
\begin{align*}
    |\overline{oc_x}| &= |\overline{oc}|\cdot\angle c_xoc\\
    &= (\sqrt{1-x^2} - k)\cdot \cos{(-x)}.
\end{align*}

Consider $|\overline{
cq_x}|$. The angle from positive $x$ direction to $cq$ is
\[\angle q_xcq = \angle q_xcm+\angle mca - \angle qca.\]


\input{PST/f1_4.tex}%?????????


We  have shown  that the angle from positive $x$ direction to $\overrightarrow{oc}$ is $-x$. Then, since $o, c$ and $m$ are co-linear, we say $\angle q_xcm = -x$ as well. Also, we know that $\angle mca = \tan^{-1}\frac{x}{k}$. 

We will calculate $\angle qca$ by analyzing arc $A$. Since $\mathcal{F}_3$ is  critical and symmetric, we have
\begin{align*}
    |A'D'DA| &=|B'B|\\
    |A|+|D| &= |B|\\
    |A| &= |B|-|D|.
\end{align*}
Then, given $|A|+|B| = |\arc{aqs}| = 2r\cdot \tan^{-1}\frac{x}{k}$ and $|D|=2|\overline{aw}| = 2x$, we have
\begin{align*}
    |A| &= |B| - |D|\\
    |A| &= |\arc{aqs}|-|A|  - |D|\\
    2|A| &= |\arc{aqs}|- |D|\\
    |A| &=\frac{|\arc{aqs}|- |D|}{2}\\
    &=\frac{2r\cdot \tan^{-1}\frac{x}{k} - 2x}{2}\\
    &=r\cdot \tan^{-1}\frac{x}{k} - x.
\end{align*}
Thus, $\angle qca = \frac{|A|}{r}$, and
\begin{align*}
    \angle q_xcq &= \angle q_xcm+\angle mca - \angle qca\\
    &= -x +\tan^{-1}\frac{x}{k}-\frac{r\cdot \tan^{-1}\frac{x}{k} - x}{r}\\
    &= -x +\tan^{-1}\frac{x}{k}-(\tan^{-1}\frac{x}{k} - \frac{x}{r})\\
    &=-x+\frac{x}{r}.
\end{align*}
Therefore, 
   \[|cq_x| = r\cdot \cos{(\angle q_xcq)}= r\cdot\cos{(-x+\frac{x}{r})}.\]
Then, 
\begin{align*}
    \frac{D_{\mathcal{F}_3}(p, q)}{2} &= |oc_x|+|cq_x|\\
    &=(\sqrt{1-x^2} - k)\cdot \cos{(-x)} +r\cdot \cos{(-x+\frac{x}{r})}.
\end{align*}
\vspace{1cm}
Hence, 
\begin{align*}
    \rho_{\mathcal{F}_3}(p, q) &=  \frac{P_{\mathcal{F}_3}(p, q)}{D_{\mathcal{F}_3}(p, q)}\\
    &=\frac{P_{\mathcal{F}_3}(p, q)/2}{D_{\mathcal{F}_3}(p, q)/2}\\
     &=\frac{\frac{2x+2r\cdot\tan^{-1}\frac{x}{k}}{2}}{(\sqrt{1-x^2} - k)\cdot \cos{(-x)} +r\cdot \cos{(-x+\frac{x}{r})}}\\
     &=\frac{x+r\cdot\tan^{-1}\frac{x}{k}}{(\sqrt{1-x^2} - k)\cdot \cos{(-x)} +r\cdot \cos{(-x+\frac{x}{r})}}\text{, as desired. }
\end{align*}
\end{proof}

\section{In the General Case}
	Consider the symmetric critical arcgons with $2n+1$ circles. Suppose the formula for $2(n-1)+1$ circles is given, the length for the shortest path should increment by $(\sqrt{r_N- x_N^2} - k_N)\cdot\cos(\tan^{-1}{\frac{x_j}{r_j}}-  \sin^{-1}{\frac{x_j}{k_j}}) $.

	For any symmetric critical arcgons with $2n + 1$ circles, 
	\[d(\vec{x}, \vec{k}) = \sum_{1\le i< n}\left(\sqrt{r_i- x_i^2} - k_i\right)\cdot\cos\left({x_1 +\sum_{1\le j< i} \tan^{-1}{\frac{x_j}{r_j}}-  \sin^{-1}{\frac{x_j}{k_j}}}\right) \]
	\[p(\vec{x}, \vec{k}) = x_1 + \sum_{1\le i< n}r_i\cdot(\tan^{-1}{\frac{x_i}{k_i}}- \sin^{-1}{\frac{x_{i+1}}{r_i}}) + r_n\tan^{-1}{\frac{x_n}{k_n}}.\]